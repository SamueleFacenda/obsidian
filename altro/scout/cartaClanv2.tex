\documentclass[12pt]{report}
\usepackage[a6paper, margin=1cm]{geometry}
\usepackage{graphicx}
\usepackage{eso-pic}
\usepackage{lmodern}
\usepackage{titlesec}

% language settings
\usepackage[T1]{fontenc}
\usepackage[italian]{babel}
\usepackage{hyphenat}
\hyphenation{mate-mati-ca recu-perare}

% font
\usepackage{newcent}

% big image command
\newcommand\BackgroundPic{%
\put(0,-50){
\parbox[b][\paperheight]{\paperwidth}{%
\vfill
\centering
\includegraphics[width=\paperwidth,height=\paperheight,%
keepaspectratio]{cover.png}%
\vfill
}}}

% paragraphs settings
\setlength{\parindent}{0cm}
\setlength{\parskip}{2mm}

% no pagenums
\pagenumbering{gobble}

\titleformat{\chapter}[display]{\centering\huge\bfseries}{\chaptertitlename~\thechapter}{5cm}{}{}% see titlesec manual
\titlespacing*{\chapter}{0pt}{0pt}{\baselineskip}%

\begin{document}
	\AddToShipoutPicture*{\BackgroundPic}
	
	\begin{titlepage}
		\centering
		\vfill
		\Huge
		CARTA DI CLAN\\
		\vskip1cm
		\large Clan Dyapason Trento 12
		
		\vfill
	\end{titlepage}
	
	
	Noi come Clan sappiamo di essere diversi da un semplice gruppo di amici, 
	dalle tante comunità di cui ognuno di noi fa parte. Ciò che ci differenzia 
	sono i valori in cui crediamo e su cui basiamo le nostre azioni, espressi 
	nella Legge e nella Promessa: fare del nostro meglio, sognare insieme e 
	puntare in alto verso un obiettivo comune, vivere il servizio, la strada e 
	tutti i momenti di comunità sempre accompagnati dalla fede. 

	La Carta di Clan è specchio della comunità e contiene i nostri sogni, i 
	nostri obiettivi e gli impegni che ci poniamo per raggiungerli. È strumento 
	di verifica del percorso del singolo e della comunità, con cui confrontarsi 
	per i Punti della Strada. 

	Vogliamo che questa Carta di Clan sia periodicamente riletta e verificata, 
	aperta a modifiche, soprattutto quando emergono nuovi obiettivi e necessità 
	o si vivono esperienze significative.

	
	\chapter*{Comunità}
	
		\section*{divisione del lavoro}
		\begin{itemize}
			\item Mettiamo a disposizione le nostre competenze per gli 
			obiettivi e i bisogni del clan.
			\item Ci suddividiamo i compiti con carichi di lavoro simili, 
			tenendo conto delle nostre capacità ed esperienze, ma 
			mettendoci in gioco per apprendere nuove competenze.
			\item Coinvolgiamo nell'organizzazione delle attività 
			anche i novizi, guidandoli nella scoperta di tradizioni 
			e modus operandi del clan.
			\item Creiamo pattuglie con un numero adeguato di persone, 
			chiarendo il compito di ciascuna e a cui non devono
			necessariamente partecipare tutti.
		\end{itemize}
		
		\section*{organizzazione delle riunioni}
		\begin{itemize}
			\item Sappiamo che la riunione è uno dei principali mezzi a 
			disposizione del clan, e che in quanto tale la sua importanza 
			non va sottovalutata.
			\item Comunichiamo le assenze e i ritardi il prima possibile, 
			così da permettere al clan di svolgere al meglio le attività. 
			\item Ogni membro è consapevole della sua importanza all'interno 
			della comunità e pertanto si impegna ad essere presente 
			il più possibile.
			\item Al fine di evitare di impiegare inutili risorse ci
			organizziamo tra di noi per i trasporti.
		\end{itemize}
		
		\section*{partecipazione}
		\begin{itemize}
			\item A riunione siamo presenti focalizzandoci sugli obiettivi
			posti, organizzando gli incontri successivi in base alle 
			necessità del clan e mettendo in primo piano gli impegni prossimi.
			\item Siamo disponibili ad assumerci impegni responsabilmente, 
			prestando attenzione alle tempistiche.
			\item Fissiamo le riunioni tenendo conto degli impegni di tutti, 
			aggiornando il quaderno di clan e comunicando sempre agli assenti 
			le decisione prese.
			\item Trattiamo tematiche che possano coinvolgere attivamente tutto 
			il clan, proponendo attività che si diversifichino tra loro.
			\item Vogliamo che la vita di clan non sia vissuta con pesantezza e 
			non che vada ad aggiungersi ai problemi di tutti i giorni.
		\end{itemize}
		
		\section*{condividere la quotidianità}
		\begin{itemize}
			\item Sappiamo di poterci fidare gli uni degli altri e vogliamo 
			creare legami autentici.
			\item Ci impegniamo a utilizzare il servizio di clan come mezzo per 
			rafforzare la comunità.
			\item Proviamo a conoscerci anche al di fuori dell’ambiente scout, 
			per consolidare i nostri legami.
		\end{itemize}
		
		\section*{correzione fraterna}
		\begin{itemize}
			\item Quando necessario adottiamo la correzione fraterna come 
			strumento per migliorare noi stessi e la comunità, 
			ma senza l'aspettativa di un riscontro immediato.
			\item Per questo ci impegniamo ad essere il più trasparenti
			possibile.
		\end{itemize}
		
		\section*{fede}
		\begin{itemize}
			\item Il momento della fede non è necessariamente di preghiera,
			ma viene vissuto dal clan in maniera differente e come un’occasione 
			di confronto.
			\item Riconosciamo  che vivere la fede singolarmente è molto più 
			complicato rispetto a quando la viviamo con la comunità. 
			\item Dovremmo dedicarci più tempo anche nella nostra quotidianità.
			\item Interpretiamo la fede come un filo conduttore, 
			che unisce tutta la nostra quotidianità e che pone le basi delle 
			nostre scelte e azioni.
		\end{itemize}
	
	\chapter*{Strada}
		
		% https://docs.italia.it/italia/designers-italia/writing-toolkit/it/bozza/suggerimenti-di-scrittura/punteggiatura-e-grammatica.html#:~:text=Usa%20il%20punto%20e%20virgola,anche%20per%20gli%20elenchi%20puntati.&text=Questa%20%C3%A8%20la%20frase%20che%20introduce%20il%20tuo%20elenco.
		\section*{per noi la strada è:}
		\begin{itemize}
			\item scoprire la bellezza del territorio e della natura, 
			pertanto cerchiamo di avere una nuova destinazione ad ogni uscita;
			\item vivere un momento di crescita personale superando gli 
			ostacoli con serenità e leggerezza, condividendo la fatica con la 
			comunità;
			\item un’occasione per confrontarci tra di noi scoprendo l'altro 
			rafforzando i nostri legami;
			\item espressione di libertà attraverso l'essenzialità;
			\item affidarsi agli altri nell'aiuto reciproco, riconoscendo i 
			propri limiti e facendo attenzione a quelli degli altri, cercando 
			di stare uniti nel cammino.
		\end{itemize}
		
		\section*{fede}
		\begin{itemize}
			\item Il clan vede Dio come parte della natura che ci circonda e di 
			cui facciamo parte.
			\item Viviamo la fede sulla strada, condividendo la stessa fatica 
			anche se in maniera differente.
			\item Vediamo la fede come un percorso di continua scoperta, 
			ponendoci sempre domande e non avendo la convinzione di essere 
			arrivati alla fine del nostro cammino.
		\end{itemize}
		
		\section*{ambiente}
		Prestiamo attenzione ad essere essenziali nell'organizzazione delle 
		uscite al fine di avere un impatto ambientale minimo (e.g. cibo di 
		stagione, trasporti ecologici, raccolta rifiuti, 
		evitare gli sprechi...).
	
	\chapter*{Servizio}
		\begin{itemize}
			\item Pensiamo che partire da un servizio comunitario, 
			guidandosi reciprocamente, possa aiutarci a viverlo singolarmente 
			con spontaneità.
			\item Riconosciamo il valore del servizio per scoprire nuove 
			realtà, affrontando con coraggio nuove sfide che ci introducano a 
			nuovi interessi.
			\item Riconosciamo la soddisfazione e la gioia che questo gesto può 
			portare, vivendo il Servizio anche come occasione di incontro.
			\item Dedichiamo parte del nostro tempo al servizio non facendolo 
			passare in secondo piano.
			\item La nostra comunità vive la fede come amore e rispetto verso 
			gli altri, nella condivisione e nel servizio.
			\item Costruiamo la nostra fede sulla condivisione dei valori dello 
			scoutismo, sentendo una fede più profonda durante il servizio. 
		\end{itemize}
		
		% bottom text
		\vspace*{\fill}
		\begin{flushright}
			% \large
			\itshape
			Firmata al Sorasass il 24 settembre 2023
		\end{flushright}

\end{document}